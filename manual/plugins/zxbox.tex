\subsection{\label{ref:ZXBox}ZXBox}
\screenshot{plugins/images/ss-zxbox}{ZXBox}{img:zxbox}
ZXBox is a port of the ``Spectemu'' ZX Spectrum 48k emulator for Rockbox
(\url{https://sourceforge.net/projects/spectemu/}).
To start a game open a tape file or snapshot saved as
\fname{.tap}, \fname{.tzx}, \fname{.z80} or \fname{.sna} in the file browser.\\
\note{As ZXBox is a 48k emulator only loading of 48k z80 snapshots is possible.}

\subsubsection{Default keys}
The emulator is set up for 5 different buttons: Up, Down, Left, Right and
Jump/Fire. Each one of these can be mapped to one key of the Spectrum Keyboard
or they can be used like a ``Kempston'' joystick. Per default the buttons,
including an additional but fixed menu button, are assigned as follows:

\begin{btnmap}
    \opt{IPOD_3G_PAD,IPOD_4G_PAD}{\ButtonMenu/\ButtonPlay/}
    \opt{RECORDER_PAD,ONDIO_PAD,IRIVER_H100_PAD,IRIVER_H300_PAD,GIGABEAT_PAD,GIGABEAT_S_PAD%
        ,IAUDIO_X5_PAD,SANSA_C200_PAD,SANSA_CLIP_PAD,SANSA_E200_PAD,SANSA_FUZE_PAD,MROBE100_PAD%
        ,PBELL_VIBE500_PAD,SANSA_FUZEPLUS_PAD,SAMSUNG_YH92X_PAD,SAMSUNG_YH820_PAD}%
        {\ButtonUp/\ButtonDown/}
    \opt{IRIVER_H10_PAD}{\ButtonScrollUp/\ButtonScrollDown/}
    \opt{IPOD_3G_PAD,IPOD_4G_PAD,RECORDER_PAD,ONDIO_PAD,IRIVER_H100_PAD%
        ,IRIVER_H300_PAD,GIGABEAT_PAD,GIGABEAT_S_PAD,IAUDIO_X5_PAD%
        ,SANSA_C200_PAD,SANSA_CLIP_PAD,SANSA_E200_PAD,SANSA_FUZE_PAD,MROBE100_PAD%
        ,IRIVER_H10_PAD,PBELL_VIBE500_PAD,SANSA_FUZEPLUS_PAD,SAMSUNG_YH92X_PAD%
        ,SAMSUNG_YH820_PAD}{\ButtonLeft/\ButtonRight}
    \opt{COWON_D2_PAD}{\TouchTopMiddle{}/\TouchBottomMiddle{}/\TouchMidLeft{}/\TouchMidRight}
    \opt{MPIO_HD200_PAD}{\ButtonVolDown / \ButtonVolUp / \ButtonRew / \ButtonFF}
    \opt{MPIO_HD300_PAD}{\ButtonRew / \ButtonFF / \ButtonScrollUp / \ButtonScrollDown}
  \opt{HAVEREMOTEKEYMAP}{& }
    & Directional movement\\
    %
    \opt{IPOD_3G_PAD,IPOD_4G_PAD,GIGABEAT_PAD,GIGABEAT_S_PAD,IAUDIO_X5_PAD%
        ,SANSA_C200_PAD,SANSA_CLIP_PAD,SANSA_E200_PAD,SANSA_FUZE_PAD,MROBE100_PAD
        ,SANSA_FUZEPLUS_PAD}{\ButtonSelect}
    \opt{RECORDER_PAD}{\ButtonPlay}
    \opt{SAMSUNG_YH92X_PAD,SAMSUNG_YH820_PAD}{\ButtonPlay{} or \ButtonFF}
    \opt{IRIVER_H100_PAD,IRIVER_H300_PAD}{\ButtonOn}
    \opt{ONDIO_PAD}{\ButtonMenu}
    \opt{IRIVER_H10_PAD}{\ButtonRew}
    \opt{COWON_D2_PAD}{\TouchCenter}
    \opt{PBELL_VIBE500_PAD}{\ButtonOK}
    \opt{MPIO_HD200_PAD}{\ButtonFunc}
    \opt{MPIO_HD300_PAD}{\ButtonEnter}
  \opt{HAVEREMOTEKEYMAP}{& }
    & Jump/Fire\\
    %
    \opt{RECORDER_PAD}{\ButtonFOne}
    \opt{ONDIO_PAD}{\ButtonOff}
    \opt{IPOD_3G_PAD,IPOD_4G_PAD}{\ButtonHold{} switch}
    \opt{IRIVER_H100_PAD,IRIVER_H300_PAD}{\ButtonMode}
    \opt{GIGABEAT_PAD,GIGABEAT_S_PAD,COWON_D2_PAD}{\ButtonMenu}
    \opt{SANSA_C200_PAD,SANSA_CLIP_PAD,SANSA_E200_PAD,MROBE100_PAD}{\ButtonPower}
    \opt{SANSA_FUZE_PAD}{Long \ButtonHome}
    \opt{IAUDIO_X5_PAD}{\ButtonPlay}
    \opt{IRIVER_H10_PAD}{\ButtonFF}
    \opt{PBELL_VIBE500_PAD}{\ButtonCancel}
    \opt{MPIO_HD200_PAD}{\ButtonRec + \ButtonPlay}
    \opt{MPIO_HD300_PAD}{Long \ButtonMenu}
    \opt{SANSA_FUZEPLUS_PAD}{\ButtonBack}
    \opt{SAMSUNG_YH92X_PAD,SAMSUNG_YH820_PAD}{\ButtonRew}
  \opt{HAVEREMOTEKEYMAP}{& }
    & Open ZXBox menu\\
\end{btnmap}

\subsubsection{ZXBox menu}
\begin{description}
\item[ Vkeyboard.]
    This is a virtual keyboard representing the Spectrum keyboard. Controls are
    the same as in standard Rockbox, but you just press one key instead of
    entering a phrase.
\item[Play/Pause Tape.] Toggles playing of the tape (if it is loaded).
\item[Save Quick Snapshot.] Saves snapshot into \fname{/.rockbox/zxboxq.z80}.
\item[Load Quick Snapshot.] Loads snapshot from \fname{/.rockbox/zxboxq.z80}.
\item[Save Snapshot.]
    Saves a snapshot of the current state. You would enter the full path and
    desired name - for example \fname{/games/zx/snapshots/chuckie.sna}. The
    snapshot format will be chosen after the extension you specified, per
    default \fname{.z80} will be taken in case you leave it open.
\item[Toggle Fast Mode.]
    Toggles fastest possible emulation speed (no sound, maximum frameskip etc.).
    This is Useful when loading tapes with some specific loaders.
\item[Options.]
    \begin{description}
    \item[Map Keys To Kempston.]
        Controls whether the \daps{} buttons should simulate a ``Kempston''
        joystick or some assigned keys of the Spectrum keyboard.
    \item[Display Speed.]Toggle displaying the emulation speed (in percent).
    \item[Invert Colours.]
        Inverts the Spectrum colour palette, sometimes helps visibility.
    \item[Frameskip]
        Sets the number of frames to skip before displaying one. With zero
        frameskip ZXBox tries to display 50 frames per second.
    \item[Sound.]Turns sound on or off.
    \item[Volume.]Controls volume of sound output.
    \item[Predefined Keymap]
        Select one of the predefined keymaps. For example \setting{2w90z} means:
        map ZXBox's \btnfnt{Up} to \setting{2}, \btnfnt{Down} to \setting{w},
        \btnfnt{Left} to \setting{9}, \btnfnt{Right} to \setting{0} and
        \btnfnt{Jump/Fire} to \setting{z}. This example keymap is used in the
        ``Chuckie Egg'' game.
    \item[Custom Keymap]
        This menu allows you to map one of the Spectrum keys accessible through the 
        plugin's virtual keyboard to each one of the buttons.
    \item[Quit.] Quits the emulator..
    \end{description}
\end{description}

\nopt{ipodvideo}{% no scaling for here, still include it?
\subsubsection{Hacking graphics}
Due to ZXBox's simple (but fast) scaling to the screen by dropping lines and
columns some games can become unplayable. It is possible to hack graphics to
make them better visible with the help of an utility such as the ``Spectrum
Graphics Editor''. Useful tools can be found at the ``World of Spectrum'' site
(\url{http://www.worldofspectrum.org/utilities.html}).}
